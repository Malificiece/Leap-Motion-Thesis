\chapter{Discussion, Future Work and Conclusion}

The purpose of this chapter is not just to reiterate what you found but rather to discuss what your findings
mean in relation to the theoretical body of knowledge on the topic and your profession. Typically, students skimp
on this chapter even though it may be the most important one because it answers the "So what?" question.

 Begin by discussing your findings in relation to the theoretical framework introduced in the literature
review. In some cases, you may need to introduce new literature (particularly with qualitative research).
 This chapter also should address what your findings mean for communication professionals in the field
being examined. In other words, what are the study's practical implications?

 Doctoral students also should discuss the pedagogical implications of the study. What does the study
suggest for mass communication education?

 This chapter next outlines the limitations of the study. Areas for future research then are proposed.
 Obviously, the thesis or dissertation ends with a brief conclusion that provides closure. A strong final
sentence should be written. 

\section{Discussion}

\subsection{Calibration}

\subsection{Reasons stylus was used instead of hands}

\subsection{Reasons that I didn't implement a word-prediction algorithm for word-gesture keyboard}

\subsection{issues with entrance/exit of touch plane}
Solve the problem of entering/leaving the touch plane (we move along a sphere) so when
	moving away from the touch plane, you often hit the key above what you're leaving on.

\section{Future Work}

\subsection{number of trials}

A greater number of trials per keyboard would have been better, in the future we need to reduce the number of keyboards each person is using so that each person can instead do something closer to 100 trials per keyboard (although 100 straight would be too tiring as shown by the first pilot study). This number can be further increased by having reoccurring visits. It would be good to have people train for each keyboard and have multiple study visits in order to get past the learning curve and get a better indication of each keyboards performance.

\subsection{tilt the leap for better tracking}

\subsection{image processing, any keyboard}

\subsection{augmented reality adaption}

\subsection{using sphere techniques to better determine the leaps position on a surface}

\subsection{with better leap detection, use hand instead of stylus}

\subsection{would word-gesture controller-based keyboard be better for consoles}

\subsubsection{Use WKG with controller}

\subsubsection{Use gesture controller (x-box kinect) for text entry}

\subsection{since we use 3 degrees, if above intersect plane, use gestures for other functions, aka controlling the mouse or other gesture related things}

\subsection{Accessibility}

\section {Conclusion}