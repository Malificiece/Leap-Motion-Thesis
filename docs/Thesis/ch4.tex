\chapter{Results and Analysis}

 This chapter addresses the results from your data analysis only. This chapter does not include discussing
other research literature or the implications of your findings.

 Usually you begin by outlining any descriptive or exploratory/confirmatory analyses (e.g., reliability tests,
factor analysis) that were conducted. You next address the results of the tests of hypotheses. You then discuss any
ex post facto analysis. Tables and/or figures should be used to illustrate and summarize all numeric information.

 For qualitative and historical research, this chapter usually is organized by the themes or categories
uncovered in your research. If you have conducted focus groups or interviews, it is often appropriate to provide a
brief descriptive (e.g., demographic) profile of the participants first. Direct quotation and paraphrasing of data from
focus groups, interviews, or historical artifacts then are used to support the generalizations made. In some cases,
this analysis also includes information from field notes or other interpretative data (e.g., life history information).

\section{Text-Entry Rate}

\section{Errors}

\subsection{MWD}

\subsubsection{not modified}

\subsubsection{modified}

\subsection{KSPC}

\subsubsection{not modified}

\subsubsection{modified}

\subsection{Total Error Rate}

\section{Fréchet Distance}

\subsection{modified vs nonmodified}

\section{Interaction with Inputs}

Use table to show the various times, distances, accuracy etc

\section{User preferences}