\chapter{Methodology --- describes/justifies data gathering methods, how data is analyzed}

Pilot 1 -- Changes for study:
Fixed:
- Constructed dictionary so inputs will all type very similar words for best comparision
- Simplified keyboard used (removed enter/space/period/shift/comma/numbers, changed location of backspace)
- keyboard mod - disallow backspace of correct letters
- keyboard mod - Removed having to press a key for confirmation of word (now confirmation on "release")
- extended keyboard view for better view of 3 dimensions for users
- limit interaction space to onscreen keyboard (works more like a phone keyboard now)
- Fix calibration order
- New Keyboard Types - Bimodal, Dynamic
- Reduced number of experiment words from 30 to 10 per input
Attempted but Unfixed:
- slight corrections to help with pinch
- Attempted Augmented Reality with Meta Glass, could not get rid of blurriness with glasses, so not implemented
Observed:
- sphere problem with touching plane
- light matters

Pilot 2 -- Changes for study:
Fixed:
- Narrowed objsectives/statistical analysis data
- Change to intermittent exit surveys
- UP IN THE AIR - remove stylus for mid air keyboards and use finger instead
- UP IN THE AIR - change controller to WGK??
- UP IN THE AIR - change dictionary builder from custom algorithm to frechet distance
- not doing - UP IN THE AIR - don't force person to be correct?? --- No need to modify (justification of why I did this)
- Made calibration faster (different depending on computer)
- Had to remove lamenent surface and only use paper, "bubbling" was interferring with detection
Observed:
- leap surface, utensil has to be perpendicular to surface for it to work properly
- sphere problem with touching plane
- light matters

Study Notes:
In some cases, the calibrated plane normal flips and the direction the tool travels is reversed.
If the plane is angled downward the touch threshold (possible culprit)??? seems to be far off the plane so the touch seems "instant".. it snaps to the touch plane and when removed it lifts high off the keyboard

 This chapter describes and justifies the data gathering method used. This chapter also outlines how you
analyzed your data.

 Begin by describing the method you chose and why this method was the most appropriate. In doing so, you
should cite reference literature about the method.

 Next, detail every step of the data gathering and analysis process. Although this section varies depending
on method and analysis technique chosen, many of the following areas typically are addressed:

--description of research design
 internal validity
 external validity

--description of population and description of and justification for type of sample used or method for
selecting units of observation

--development of instrument or method for making observations (e.g., question guide, categories for content
analysis)
 pre-test
 reliability and validity of instrument or method

--administration of instrument or method for making observations (e.g., interviews, observation, content
analysis)

--coding of data

--description of data analysis
 statistical analysis and tests performed
 identification of themes/categories (qualitative or historical research) 

\section{Study Design === Task???}

--- This should really be a description of the exact task performed (the number of trials, the number of words, the number of input devices) and what they were trying to do, the procedure doesn't include this, it just talks about the experiment ordering as a whole.

\subsection{Pilot Study 1}

\subsection{Pilot Study 2}

\subsection{Final Study}

HINDSIGHT:
A greater number of trials per keyboard would have been better, in the future we need to reduce the number of keyboards each person is using so that each person can instead do something closer to 100 trials per keyboard. This number can be further increased by having reoccuring visits. It would be good to have people train for each keyboard and have multiple study visits in order to get past the learning curve and get a better indication of each keyboards performance.

The task performed by the subjects consisted of 15 trials for each of the 6 keyboard input devices giving a total of 90 trials per subject. For each trial, a word is selected at random from a predetermined dictionary of 15 words corresponding to the active keyboard input and appears on the screen. This process is repeated until all of the words in the dictionary have been selected. TODO: See section DICTIONARY_EXPLANATION for details on how the predetermined dictionaries were generated. Once a word appears on the screen, subjects were required to correctly type in the displayed word using the active keyboard. During the gesture-typing process, correctly matching letters were highlighted in green and errors are highlighted in red. The user was required to use the active keyboard's backspace key to remove errors. Correct letters in a word are protected from the backspace key 

\section{Study Setup === Input Devices and Interaction}

Computer Used, Leap, Tech etc, the table touch screen.

\section{Participants}

How many subjects we got for each phase, some identifiers, male/female/hours used/experience etc. Describe how all subjects participated in every keyboard and all used the same dictionary to more accurately depict each of the keyboard inputs.

A sample size of 14 was used in this study. The justification for this sample size comes from the formula to calculate the sample size for two independent group means using a pooled standard deviation:

\begin{equation}
 N = \frac{2(z_{\frac{\alpha}{2}} + z_{1-\beta})^2}{(\frac{\mu_1 - \mu_2}{\sigma_{pooled}})^2}
\end{equation}

Power of $1-\beta = 0.80$ and a significance level of $\alpha = 0.05$ were used when calculating the sample size. See Appendix G to see what variables and keyboard comparisons were used in the sample size calculation. The derived sample size was the average sample size for all relevant variable comparisons based on the study objectives. Outliers requiring a sample size greater than 100 were removed. Furthermore, a sample size of 14 justifies the Latin Squares design for 7 input methods. The Latin Squares design was chosen for counterbalancing the experimental design and to reduce the effect of participation in one condition affecting performance of other conditions. Further details are explained in Section \ref{3_experimental_design}, \textit{Experimental Design}.

\subsection{Pilot Study 1}

\subsection{Pilot Study 2}

\section{Procedure}

The process of how we went through the steps (this is the same procedure as in the protocol, so this shoudl be an easy copy and paste). Just make it more beefy and change the vocabulary to reflect the trails/tasks.

There will only be a single study visit for each subject in which all experiments will be performed and will end with an exit survey. The study visit will take no more than 60 minutes to complete. The subject will be asked to do the following procedures:

\begin{itemize}
\item Complete a set of tasks on the computer for each of the 7 virtual keyboard inputs. These tasks, for all of the virtual keyboard inputs, are expected to take an upward bound of 54 minutes to complete. For each of the 7 virtual keyboard inputs, the subject will complete the following steps:
\begin{itemize}
\item Given a brief explanation of the current input. This explanation will take a total of about 30 seconds. The dialog will be similar to: “This is the ABC keyboard (e.g., standard, controller, leap-air, leap-surface, etc.). It is a JKL keyboard (e.g., mid-air, controller-based, or touch-based) and you will use XYZ (e.g., stylus, hand, or controller) to interact with it.” The subject will then be given the interaction object and allowed to interact with the virtual keyboard input.
\item The subject will be given multiple opportunities to optionally recalibrate the keyboard interaction-space (if applicable to the current input) as many times as needed. Some inputs do not have an interaction-space and therefore do not require calibration of any kind. This task will take a total of about 10 seconds for each calibration and is expected to take no more than 6 calibrations (1 minute) for each applicable input.
\item The subject will then be given the opportunity to use each virtual keyboard input to type in a variety of practice words. Practice words are randomly selected from a large dictionary but filtered to remove swear words and words used in the experiments. The subject will be able to attempt as many practice words as needed until they feel comfortable with the current virtual keyboard input. At any time during this phase, the participant can opt to recalibrate the interaction-space if applicable. This task is expected to take a maximum time of 3 minutes.
\item Next, the subject will be going through the experiment. They will type in a total of 10 words for the current virtual keyboard input. These words are preselected words for each input before the experiment begins and have been selected based on a similarity calculation and filtered for swear words. This task is expected to take a maximum time of 3 minutes.
\item Finally, there will be a small survey section after using the current keyboard input to rate each one on the Likert scale relating to difficulty, discomfort and fatigue experienced when using the devices. This task is expected to take a maximum time of 30 seconds. See Appendix D for the keyboard input survey example.
\end{itemize}
\item After all experiments are completed for each input device, the subject will be asked to fill out an exit survey. This exit survey will obtain basic data such as age, gender, major, and handedness as well as several questions about any prior experience or impairments that might relate to the study. Finally, the exit survey will have a section to rank each device on a numerical scale. This exit survey will take a total of about 5 minutes. See Appendix E for the full exit survey.
\end{itemize}

\begin{table}[h]
\centering
\caption[Schedule of Assessments]{\centering Schedule of Assessments for a single study visit (in minutes).}
\label{table_schedule_of_assessments}
\resizebox{\textwidth}{!}{\begin{tabular}{ l | c c c c c c c | c}
    \hline
    {} & Controller & Touch Screen & Leap-Surface & Leap-Air & Leap-Air & Leap-Air & Leap-Air & \textbf{total}  \\
    {} & {} & {} & {} & Static & Dynamic & Pinch & Bimodal & {}  \\
    \hline
    explain & .5 & .5 & .5 & .5 & .5 & .5 & .5 & \textbf{3.5} \\
    calibrate & 0 & 0 & 1 & 1 & 1 & 1 & 1 & \textbf{5} \\
    practice & 3 & 3 & 3 & 3 & 3 & 3 & 3 & \textbf{21} \\
    task & 3 & 3 & 3 & 3 & 3 & 3 & 3 & \textbf{21} \\
    survey & .5 & .5 & .5 & .5 & .5 & .5 & .5 & \textbf{3.5} \\
    \hline
    \textbf{total} & \textbf{7} & \textbf{7} & \textbf{8} & \textbf{8} & \textbf{8} & \textbf{8} & \textbf{8} & \textbf{54} \\
    \hline
\end{tabular}}
\end{table}

\subsection{Pilot Study 1}

The number of words that were used in the initial pilot study were 30 words per keyboard input totalling 210 trials per subject. This lead to complaints about fatigue, one subject completely giving up on the Leap-Air Static keyboard input. The number of words per keyboard input was reduced to 10 words.

\subsection{Pilot Study 2}

During the second pilot study subjects were asked to fill out the Likert scale, keyboard evaluations for discomfort, ease of use and fatigue at the end of the experiment during the final exit survey. Subjects were having trouble recalling which keyboard was which so this prompted a change to the procedure to include intermittent keyboard evaluation surveys after each keyboard input task was completed rather than during the final exit survey.

\section{Experimental Design} \label{3_experimental_design}

TODO: get reference for latin squares design, Alivin has a good one in 4.4.6 Design section. Also get a reference for within-subjects design so that we can have a reference for these statements in the next paragraph.

A Within-Subjects design was used for this study. The strength of a Within-Subjects design is that the overall power will increase and there will be a reduction in error variance associated with individual differences. The weakness of using the Within-Subjects design is that it suffers from carryover effects between each keyboard input device. The participation in one condition may affect performance in other conditions. To account for this weakness, the study was supplemented with a Latin Squares design for counterbalancing. Table \ref{table_latin_squares_rep_1} and Table \ref{table_latin_squares_rep_2} show how the Replicated Latin Squares design was utilized for 7 different keyboard inputs and a sample size of 14.

\begin{table}[h] % b - for bottom; !t - for top
\centering
\caption[Latin Squares Design Rep 1]{\centering First rep of a Replicated Latin Squares design for 14 subjects and 7 conditions.}
\label{table_latin_squares_rep_1}
\begin{tabular}{c | c c c c c c c}
    \hline
    \multicolumn{8}{c}{Rep 1} \\
    subjects & \multicolumn{7}{c}{conditions} \\
    \hline
    1 & A & B & C & D & E & F & G \\
    2 & B & C & D & E & F & G & A \\
    3 & C & D & E & F & G & A & B \\
    4 & D & E & F & G & A & B & C \\
    5 & E & F & G & A & B & C & D \\
    6 & F & G & A & B & C & D & E \\
    7 & G & A & B & C & D & E & F \\
    \hline
\end{tabular}
\end{table}

\begin{table}[h] % b - for bottom; !t - for top
\centering
\caption[Latin Squares Design Rep 2]{\centering Second rep of a Replicated Latin Squares design for 14 subjects and 7 conditions.}
\label{table_latin_squares_rep_2}
\begin{tabular}{c | c c c c c c c}
    \hline
    \multicolumn{8}{c}{Rep 2} \\
    subjects & \multicolumn{7}{c}{conditions} \\
    \hline
    8 & G & A & B & C & D & E & F \\
    9 & A & B & C & D & E & F & G \\
    10 & B & C & D & E & F & G & A \\
    11 & C & D & E & F & G & A & B \\
    12 & D & E & F & G & A & B & C \\
    13 & E & F & G & A & B & C & D \\
    14 & F & G & A & B & C & D & E \\
    \hline
\end{tabular}
\end{table}

\subsection{Pilot Study 1}

The initial pilot study used a Within-Subjects design without any counterbalancing.

\subsection{Pilot Study 2}

The second pilot study also used the Within-Subjects design and introduced the Latin Squares design for counterbalancing. There were only 7 subjects in the second pilot study; therefore, a Replicated Latin Squares design was not used. See Table \ref{table_latin_squares_no_reps} for a standard Latin Squares design.

\begin{table}[h] % b - for bottom; !t - for top
\centering
\caption[Latin Squares Design with No Replications]{\centering Latin Squares design for 7 subjects and 7 conditions.}
\label{table_latin_squares_no_reps}
\begin{tabular}{c | c c c c c c c}
    \hline
    subjects & \multicolumn{7}{c}{conditions} \\
    \hline
    1 & A & B & C & D & E & F & G \\
    2 & B & C & D & E & F & G & A \\
    3 & C & D & E & F & G & A & B \\
    4 & D & E & F & G & A & B & C \\
    5 & E & F & G & A & B & C & D \\
    6 & F & G & A & B & C & D & E \\
    7 & G & A & B & C & D & E & F \\
    \hline
\end{tabular}
\end{table}

\section{Dependent Measures === Breaking the Standard Evaluation Method}

Why did I make the changes to the variables I used, or why did I use the task that I used. (Forcing backspace/correctness and not allowing correct stuff to be backspaced etc. Should be justification from Pilot study + andy sources on this kind of stuff).

The statistical methods being used will consist of One-Way ANOVAs for each set of the recorded and calculated variables for the quantitative data. There will be 7 conditions for each which are representative of the 7 separate inputs methods being tested. Then, Tukey's HSD (honest significant difference) for multiple-compare will be used in conjunction with the ANOVAs using a post-hoc analysis. In addition, a Two-Way ANOVA will be used on each set of variables in conjunction with the previous swipe experience variable.

The qualitative approaches being analyzed are the Likert scale and a ranking system. The Likert scale will use a One-Way ANOVA for analysis because responses to several Likert questions may be summed up providing that all the questions use the same Likert scale and that the scale is a defensible approximation to an interval scale. The ranked system will use the Friedman's test for analysis, and will be subject to a post-hoc analysis using Tukey's HSD.


\section{Word Dictionary}

\subsection{Why Similar Words are Important}

\subsection {Word Dissimilarity Algorithm}

\section{Word-Gesture Keyboard Construction}

\subsection{Leap Motion --- Needs to be referenced in intro}

\subsection{Writing the Code --- rename}

\subsection{Designing various WGKs}

\subsubsection{Separating words --- rename (figure out ordering)}

\subsubsection{Separation of motor space and display space --- rename (figure out ordering)}

\subsubsection{Size of motor space / User calibrated --- rename (figure out ordering)}

\subsection{Word-Gesture Recognition}

Lack of recognition and reason for it. Simulated recognition.

\section{WGKs}

\subsection{Controller WGK --- Need to see if we want to change the implementation to a Controller WGK from std controller}

\subsection{Surface WGK}

\subsubsection{Touch screen}

\subsubsection{leap surface emulation}

\subsection{Mid-Air WGKs}

\subsubsection{static}

\subsubsection{dynamic}

\subsubsection{bimodal}

\subsubsection{pinch gesture emulation}

\section{Study/Experimental Design}
