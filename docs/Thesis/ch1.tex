\chapter{Introduction}
With the increase in gesture-controlled interfaces for touch screen and other modern devices, gesture-controls have started to see a transition for use in mid-air. Mid-air, gesture-controlled content has seen its emergence in large displays \cite{ref_pan_zoom_large_dispalys,ref_large_screen_pointing_gestures}, smart phones \cite{ref_multiscale_navigation}, augmented reality \cite{ref_augmented_reality}, and desktop computers \cite{ref_leap_painting,ref_darren_thesis,ref_alvin_thesis,ref_leap_pointing_device}. Mid-air pointing has been a common approach to many of these gesture-controlled interactions and is used to select and manipulate on-screen objects \cite{ref_large_display_pointing,ref_air_pointing,ref_ray_pointing_large_displays,ref_shadow_reaching,ref_freehand_pointing_large_displays,ref_large_screen_pointing_gestures}; however, means of reasonable mid-air text-entry are fairly new. Past technologies allowed for mid-air text-entry, but those approaches had fallen short of any meaningful text-entry rates \cite{ref_visual_touchpad}. More modern approaches of mid-air text-entry had seen improved results but still low, around 13 \cite{ref_selection_based_mid_air} to 18.9 \cite{ref_mid_air_text_large_displays} words per minute. These approaches were limited to selection of individual characters and also lacked the multi-tap feature of touch-based entry \cite{ref_selection_based_mid_air,ref_airstroke,ref_mid_air_text_large_displays}. Last year, the largest improvement was seen in mid-air text-entry when Markussen et al. \cite{ref_vulture} transitioned word-gesture keyboards for mid-air use with the development of Vulture, reaching a text-entry rate of 20.6 words per minute for their initial study. They achieved an even higher text-entry rate of 28.1 Words Per Minute in a second study with training of single, short phrases, indicating that learning the new techniques will help bring mid-air text-entry closer to touch-based text-entry. Vulture reached 59\% of the text entry rate of touch-based inputs \cite{ref_vulture}.

The purpose of this study was to use the Leap Motion \cite{ref_leap_motion}, a new and emerging technology \cite{ref_leap_painting,ref_leap_device_evaluation_1,ref_leap_device_evaluation_2}, for interpreting mid-air gestural inputs for text entry \cite{ref_leap_tech}. The only previous attempt for text entry with the Leap Motion was in mid-air handwriting \cite{ref_air_handwriting}; however, even regular hand-writing is slow and confined around 15 words per minute \cite{ref_handprinting_alternatives}. Instead, this study aims to follow the path of Markussen et al. \cite{ref_vulture} and use the Leap Motion to extend word-gesture keyboards to mid-air text-entry. Word-gesture keyboards have garnered popularity with the advent of smart phones and tablets and have been proven to perform well on touch screens \cite{ref_shape_writing,ref_the_word_gesture_keyboard,ref_shapewriter_iphone}. This study intends to use the Leap Motion to find alternatives to mid-air, word-gesture keyboards and find a better approach than wearing a glove or detecting pinching \cite{ref_vulture,ref_airstroke} for the mid-air equivalent of tapping and releasing for delimiters of words. This study will explore the option of using the extra degrees of freedom available in mid-air (e.g., depth) which was decided against by Markussen et al. \cite{ref_vulture} to point out the problems of solutions of these techniques. This study will also use other techniques of simulating touch for the mid-air equivalent of tapping and releasing for delimiters of words. The idea was to achieve something that was as simple as using touch-based devices while still allowing for a multi-functional workspace.

Touchless gesture-controllers for mid-air text-entry will benefit augmented reality (e.g., Google Glass, Microsoft HoloLens) and virtual reality (e.g., Oculus Rift). Furthermore, they will greatly benefit the medical industry (e.g., operating rooms, those with disabilities, pathogen research), providing sanitary, sterile, and efficient means of text-entry in fragile environments.

\section{Motivation}
% TODO: Need to find references about the inefficiency of keyboards for the elderly or those with disabilites
% As well as tie in some of the refs from above.
This research was inspired by the older generation, especially veterans who are amputees, and those who have trouble typing efficiently with standard computer keyboards. It is a arduous and painstaking task to type long emails or messages when using one hand, or an improvised device for double amputees, and having to type each character individually. This led to seeking a better method for text-entry, focused on being able to be used by those who are generally limited to single-input text-entry when using a standard keyboard. A method to make typing and other interactions with computers simpler and more efficient, not as tedious as they are for users restricted to single-input text-entry. Observing the work of Alvin Jude Hari Haran and Darren Guiness being done on accessibility with using mid-air gestures to control the mouse using the Leap Motion \cite{ref_leap_motion,ref_alvin_thesis,ref_darren_thesis}, influenced this research to put the same technology to use for working with mid-air keyboards. It was immediately evident with all of the phones and tablets in the market today, coming standard with word-gesture keyboards and efficient, single-handed text-entry, that it would be beneficial to apply word-gesture keyboard techniques to mid-air to replace conventional keyboard use.

At the beginning of this research, there were no other studies on mid-air, word-gesture keyboards. This thesis sought to try many different techniques, focused mostly on word-gesture keyboards, to explore their usage in mid-air. Shortly after research began, Vulture was released \cite{ref_vulture}, an exciting study, and the first to apply word-gesture keyboards to mid-air text-entry; as a result, the focus of this research changed from seeing \textbf{\textit{if}} word-gesture keyboards could work for mid-air text-entry to exploring other alternatives for \textbf{\textit{how}} they are interacted with.