\chapter{Introduction}
\hspace{\parindent}With the increase in gesture-controlled interfaces for touch screen and other modern devices, gesture-controls have started to see a transition for use in mid-air (i.e., screen-less interaction). Mid-air, gesture-controlled content has seen its emergence in large displays \cite{ref_pan_zoom_large_dispalys,ref_large_screen_pointing_gestures}, smart phones \cite{ref_multiscale_navigation}, augmented reality \cite{ref_augmented_reality}, and desktop computers \cite{ref_leap_painting,ref_darren_thesis,ref_alvin_thesis,ref_leap_pointing_device}. Mid-air pointing has been a common approach to many of these gesture-controlled interactions and is used to select and manipulate on-screen objects \cite{ref_large_display_pointing,ref_air_pointing,ref_ray_pointing_large_displays,ref_shadow_reaching,ref_freehand_pointing_large_displays,ref_large_screen_pointing_gestures}. However, techniques for reasonable mid-air text-entry rates are fairly new. Past technologies allowed for mid-air text-entry (e.g., Touchpad, Airstroke), but those approaches had fallen short of any meaningful text-entry rates \cite{ref_visual_touchpad,ref_airstroke}. More modern approaches of mid-air text-entry have seen improved results but still low, around 13.2 \cite{ref_selection_based_mid_air} to 18.9 \cite{ref_mid_air_text_large_displays} words per minute. These approaches were limited to selection of individual characters and, at best, similar to the multi-tap features of touch-based entry \cite{ref_selection_based_mid_air,ref_airstroke,ref_mid_air_text_large_displays}. Last year, the largest improvement was seen in mid-air text-entry when Markussen et al. \citeyear{ref_vulture} transitioned word-gesture keyboards to mid-air with the development of Vulture, reaching a text-entry rate of 20.6 words per minute for their initial study. They achieved an even higher text-entry rate of 28.1 Words Per Minute in a second study with training of single, short phrases, indicating that learning the new techniques will help bring mid-air text-entry closer to touch-based text-entry. Vulture reached 59\% of the text-entry rate of touch-based inputs which is the fastest mid-air text-entry rate seen yet \cite{ref_vulture}.

The purpose of this thesis was to use the Leap Motion controller \cite{ref_leap_motion}, a new and emerging technology \cite{ref_leap_painting,ref_leap_device_evaluation_1,ref_leap_device_evaluation_2}, that interprets mid-air gestural inputs, for text entry \cite{ref_leap_tech}. The only previous attempt for text entry with the Leap Motion controller was in mid-air handwriting \cite{ref_air_handwriting}. However, even analog handwriting is slow and confined to around 15 words per minute \cite{ref_handprinting_alternatives}. Instead, this study aims to follow the path of Markussen et al. \cite{ref_vulture} and use the Leap Motion controller to extend word-gesture keyboards to mid-air text-entry. Word-gesture keyboards have garnered popularity with the advent of smart phones and tablets and have been proven to perform well on touch screens \cite{ref_shape_writing,ref_the_word_gesture_keyboard,ref_shapewriter_iphone}. This study intended to use the Leap Motion controller to find alternatives to word separation for text-entry mid-air, word-gesture keyboards and find a more natural approach to wearing a glove or detecting pinching \cite{ref_vulture,ref_airstroke}, for the mid-air equivalent of tapping and releasing for delimiters of words. This study explored using the extra degrees of freedom available in mid-air (e.g., depth) to examine the problems and solutions to these techniques. This study also used other techniques of simulating touch (e.g., gesture prediction, bimodal input) for the mid-air equivalent of tapping and releasing for delimiters of words. The idea was to achieve something that was as simple as using touch-based devices while still allowing for a multi-functional workspace.

Touchless gesture-controllers for mid-air text-entry will benefit augmented reality (e.g., Google Glass, Microsoft HoloLens) and virtual reality (e.g., Oculus Rift). Furthermore, they will greatly benefit the medical industry (e.g., operating rooms, those with disabilities, pathogen research), providing sanitary, sterile, and efficient means of text-entry in fragile environments.

\section{Motivation}
This research was inspired by the older generation, especially veterans who are amputees \cite{ref_amputee_EMG_keyboard}, and those who have trouble typing efficiently with standard computer keyboards \cite{ref_aging_and_technology,ref_review_computer_input_devices_older_users}. Typing long emails or messages when using one hand, or an improvised device for double amputees, and having to type each character individually is an arduous and painstaking task. This led to seeking a better method for text-entry, focused on usability for those who are generally limited to single-input text-entry when using a standard keyboard; a method to make typing and other interactions with computers simpler and more efficient, not as tedious as they are for users restricted to single-input text-entry. Observing the work of Alvin Jude Hari Haran and Darren Guiness on accessibility with using mid-air gestures to control the mouse using the Leap Motion controller \cite{ref_leap_motion,ref_alvin_thesis,ref_darren_thesis} influenced this research to put the same technology to use with mid-air keyboards. With all of the phones and tablets in the market today shipping standard with word-gesture keyboards and efficient, single-handed text-entry, it was immediately evident that it would be beneficial to apply word-gesture keyboard techniques to mid-air to replace conventional keyboard use.

At the beginning of this research, there were no other studies on mid-air, word-gesture keyboards. This thesis sought to try many different techniques, focused mostly on word-gesture keyboards, to explore their usage in mid-air. Shortly after research began, Vulture was released \cite{ref_vulture}, an exciting study, and the first to apply word-gesture keyboards to mid-air text-entry. As a result, the focus of this research changed from seeing \textit{if} word-gesture keyboards could work for mid-air text-entry to exploring other alternatives for \textit{how} they can work.

\section{Thesis Overview}
This thesis was completed in compliance with the requirements for the Master of Science in Computer Science at Baylor University. To begin, background research was conducted on mid-air pointing, mid-air text-entry, word-gesture keyboards, and text-entry evaluation, which are presented in Chapter~\ref{literature_review}. With this understanding, Chapter~\ref{keyboard_design} discusses the design and construction of this study's different word-gesture keyboards using a finger or stylus as input. The word-gesture keyboard implementations and keyboard layouts were heavily influenced by the initial pilot studies presented in Chapter~\ref{pilot_studies}. Chapter~\ref{methodology} details the task design, procedure, and dependent measures focused on in the full study while also highlighting a few challenges and limitations and their possible solutions. Results of the the various word-gesture keyboards are presented and discussed in depth in Chapter~\ref{results}. Discussion of the findings, potential future work, and concluding remarks are given in Chapter~\ref{conclusion}.