\chapter{Introduction --- purpose for study, significance of study, original contribution}

In the first chapter, clearly state what the purpose of the study is and explain the study's significance. The
significance is addressed by discussing how the study adds to the theoretical body of knowledge in the field and the
study's practical significance for communication professionals in the field being examined.

Ph.D. students also must explain how their research makes an original contribution to the body of
knowledge in their discipline. They also should address the significance of the study for mass communication
education.

It is especially critical that this chapter be well developed. Without a clearly defined purpose and strong
theoretical grounding, the thesis or dissertation is fundamentally flawed from the outset. 

\section{Motivation}

With the increase in gesture-controlled interfaces for touch screen and other modern devices, gesture-controls have started to see a transition for use in mid-air. Mid-air, gesture-controlled content has seen it's emergence in large displays [10], smart phones [4], augmented reality [12], and desktop computers [15]. Mid-air pointing has been a common approach to many of these gesture-controlled interactions and is used to select and manipulate on-screen objects [1, 2, 5, 14, 18]; however, means of reasonable mid-air text entry are fairly new. Past technologies allowed for mid-air text entry, but those approaches have fallen short of any meaningful text entry rates [7]. More modern approaches of mid-air text entry have seen improved results but still low, around 13 [8] to 18.9 [13] words per minute. These approaches were limited to selection of individual characters and also lacked the multi-tap feature of touch-based entry [8, 11, 13]. Last year, the largest improvement was seen in mid-air text entry when Markussen et al. [9] transitioned word-gesture keyboards for mid-air use with the development of Vulture, reaching a text entry rate of 20.6 words per minute for their initial study. They achieved an even better text entry rate of 28.1 words per minute in a second study, with training and repeated measures, indicating learning the new techniques will help bring mid-air text entry closer to touch-based text entry. Vulture reached 59% of the text entry rate of touch-based inputs [9].

The purpose of this study is to use the Leap Motion, a new and emerging technology [15], for interpreting mid-air gestural inputs for text entry [16]. The only previous attempt for text entry with the Leap Motion was in mid-air handwriting [17]; however, even regular hand-writing is slow, and confined around 15 words per minute [3]. Instead, this study aims to follow the path of Markussen et al. [9] and use the Leap Motion to extend word-gesture keyboards to mid-air text entry. Word-gesture keyboards have garnered popularity with the advent of smart phones and tablets and have been proven to perform well on touch screens [6, 19, 20]. This study intends to use the Leap Motion to find alternatives to mid-air, word-gesture keyboards and find a better approach than wearing a glove or detecting pinching [9, 11] for the mid-air equivalent of tapping and releasing for delimiters of words. This study will explore the option of using the extra degrees of freedom available in mid-air (e.g., depth) which was decided against by Markussen et al. [9] to create virtual keyboards in mid-air, and it will also use several techniques of simulating touch for the mid-air equivalent of tapping and releasing for delimiters of words. The goal is to make it feel as similar to using a touch-based device as possible while still allowing for common gestures.

The rationale behind this research is to improve mid-air text entry using techniques that will still allow for other gesture-controls when working with gesture-interfaces [1, 2, 5, 14, 18]. Touch-less gesture-controllers with mid-air text entry will benefit augmented reality (e.g., Google Glass, Microsoft HoloLens) as well as benefit the medical world (e.g., operating rooms) when it comes to sanitation and sterile environments to reduce the spread of pathogens.

\section{Definition}

\section{Gestures/Text Entry: Past, Present and Future --- rename this and subsections (Related Works)}

\subsection{A Brief History}

\subsection{Current Trends}

\subsection{Future Plans}

\section{Problems, Challenges, Limitations --- rename this section}

\section{Solution Scope --- rename}

1.	Explore alternatives using the Leap Motion for mid-air, word-gesture keyboards to find a better approach than pinching for the mid-air equivalent of tapping and releasing for delimiters of words and increasing text entry rates [9].

2.	Explore alternatives using the Leap Motion to touch-based, word-gesture keyboards. Alternatives, without training or repeated measures, are not expected to surpass touch-based, word-gesture keyboards in text entry rates; however, some alternatives are expected to be similar in error rates, precision, and usability.

3.	Look for evidence of a correlation between having previous word-gesture keyboard experience and the text entry rates, error rates, precision, and usability of the various word-gesture keyboards.
